\section{Mecanismo para Recordes}
Os recordes das Actividades estão organizados da seguinte forma: Todas as actividades implementam ObjectRecord, e em cada actividade tem uma enumeração de Records e outra com os EnumAttr.
Os Records e os EnumAttr devem ser acedidos através dos métodos da interface ObjectRecord.

Quanto uma actividade extende a outra (Exemplo: Altimetry extende a Distance), os métodos de acesso aos Records do Object funcionam de tal forma que se o recorde estiver definido naquela Actividade encontrou e pára, caso contrario pesquisa no nível seguinte.

\subsection{Record interface}
\label{sec:interface}
A interface Record representa um tipo de recorde, é caracterizado por:
\begin{itemize}
 \item Um Nome e um ID. 
 \item Um atributo principal, que é atributo que irá ser medido.
 \item Um atributo fixo (opcional), que funciona como constante para criar recordes semelhantes mas com um valor diferente. (Exemplo: \"recorde de tempo em X metros\", X pode é visto como um atributo fixo)
\end{itemize}
Está interface também oferece a funcionalidade de verificar, quando necessário, se um valor de um atributo é semelhante ao valor do atributo fixo

\subsection{EnumAttr interface}
\label{sec:interface}
A interface EnumAttr representa um atributo, é caracterizado por um Nome e um ID. Esta interface permite aceder aos atributos de forma genérica ...

\subsection{ObjectRecord interface}
\label{sec:interface}
A interface ObjectRecord representa uma entidade constituída por varias recordes, assim um objecto que implemente ObjectRecord terá as seguintes funcionalidades:

\begin{itemize}
 \item Porta de acesso para os recordes através do seu ID.
 \item Porta de acesso para os atributos através do seu ID.
 \item Ser comparado com outro ObjectRecord para um determinado Record, e saber qual é o melhor.
 \item Representar este ObjectRecord para um determinado Record sobe a forma de String.
\end{itemize}







%%%%%%%%%%%%%%%%%%%%%%%%%%%%%%%%    %%%%%%%%%%%%%%%%%%  MANANGERRRRR

\section{DataSet}
\label{sec:manager}
A class Dataset é o conjunto de toda a informação do referente à aplicação criada, contendo um conjunto de utilizadores e outro de eventos. Esta informação é acedida através da class manager.



\section{Encapsulamento: Manager}
\label{sec:manager}
A class Manager foi criada para facilitar o encapsulamento de Map's ou Set's, e realiza as operações básicas de gestão (na adição faz um clone, na pesquisa retornar um clone)

Quando se instância a class Manager, passa-se a estrutura que pretendemos encapsular, a partir dai esta estrutura nunca mais pode ser acedida. Portanto caso seja necessária para outra algum método especifico da estrutura, antes de instanciar deverá ser guardada.

Assim torna-se suficiente retornar o Manager para que o exterior consiga aceder de forma encapsulada à informação.

Para poder utilizar um Manager de Map's os objetos devem implementar a interface ObjectKey. O valor retornado por essa função será a Key do objecto no map.

